\documentclass{beamer} % use [handout] option to stop pauses

\usepackage[utf8]{inputenc}
\usepackage{preamble}

\title{Type Theory and Homotopy}
\subtitle{I. Constructions and Dependence}
\author{
  G. A. Kavvos \inst{1}
  \and
  Alex Kavvotsky \inst{2}
}
\institute{
  \inst{1} University of Bristol
  \inst{2} Technological University of Elbonia
}
\titlegraphic{
  \includegraphics[scale=0.1]{Bristol.png}
}
\date{Panhellenic Logic Symposium, 6--10 July 2022}


\begin{document}


% title page; necessary.
\frame{\titlepage}


\section{Intuitionism and Constructions}

\begin{frame}
  \frametitle{Intuitionism, Constructivism, and Type Theory}
  
  \begin{itemize}
    \item Many different philosophies: Brouwerian intuitionism, Heyting
      arithmetic, Russian constructivism, Bishop-style mathematics, etc. (see
      Stanford Encyclopedia of Philosophy entries)

    \item One common feature:
      \begin{center}
        \shadowbox{
          \begin{minipage}{0.70\textwidth}
            \centering
            To prove that a mathematical object exists\\
            you must show how to construct it.
          \end{minipage}
        }
      \end{center}
      
    \item In particular, the details of the construction matter.
      
    \item Modern algebra: the structure of an isomorphism matters.

    \item \textbf{Martin-L\"of Type Theory} (MLTT) was created as a
      formalization of Bishop-style constructive mathematics.
      
    \item Less focus on \textbf{truth}, more focus on \textbf{proof}.

    \item The \textbf{law of the excluded middle} (LEM) $\phi \lor \lnot \phi$
      is rejected.
  \end{itemize}
\end{frame}

\begin{frame}
  \frametitle{Two-column slide}
  \begin{columns}
  \column{0.5\textwidth}
  This is a text in first column.
  $$E=mc^2$$
  \begin{itemize}
  \item First item
  \item Second item
  \end{itemize}
  
  \column{0.5\textwidth}
  This text will be in the second column
  and on a second thoughts, this is a nice looking
  layout in some cases.
  \end{columns}
  \end{frame}
  
\begin{frame}
  \begin{center}
    \begin{tabular}{l|c|l}
      1 & \only<1>{2a} \only<2>{much longer 2b} & 3
    \end{tabular} 
  \end{center}
\end{frame}



\begin{frame}
  \frametitle{References}
  \bibliographystyle{amsalpha}
  \bibliography{refs.bib}
  \nocite{martin-lof_1975}
\end{frame}

\end{document}